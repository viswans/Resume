\documentclass{mycv2page}
\author{Sudharshan.V}
\RollNumber{EE08B049}
\Address{321, Godavari Hostel, IIT Madras}
\PhoneNumber{+91 94454 53803}
\Email{sudharshan.visu@gmail.com}

\begin{document}
\maketitle
\section{Education}
\begin{EducationTable}
	\EduDetails{\textbf{B.Tech}:Electrical Engineering \textbf{Minor}:Physics 
   \newline \textbf{M.Tech}:VLSI and Microelectronics 
   }{IIT Madras, Chennai}{\textbf {9.05} }{2013}
  \EduDetails{12th Grade}{P.S.B.B. K.K.Nagar, Chennai}{95.0}{2008}
  \EduDetails{10th Grade}{P.S.B.B. K.K.Nagar, Chennai}{90.8}{2006}
\end{EducationTable}

\section{Scholastic Achievements}
\Point{
\textbf{Winner} of \textbf{TAU Contest 2012-13}. Implemented a multi-threaded Statistical Timing Tool for multi-core CPU architecture. \textit{Team Size: 2}}
\Point{
\textbf{Selected} to represent \textbf{Electrical Engineering dept. IIT Madras} in the \textbf{Indo-German Winter Academy} conducted in December 2011 by IIT Delhi-FAU Erlangen. Presented a Technical Talk on \textbf{Resistive Memory Devices}}
\Point{
\textbf{Selected} to the \textbf{International Olympiad For Informatics Training Camp}. Ranked in \textbf{top 25} at the Indian National Olympiad For Informatics(\textbf{INOI})}
\Point{
\textbf{First} in the Physics Talent Test in January 2008 conducted by The Physics Society, Chennai among \textbf{618} students}
\Point{
Recipient of \textbf{Central Scheme Sector Scholarship} sponsored by the \textbf{MHRD} for excelling in XII CBSE examinations}

\section{Projects}
\datedsubsection{Final Year Project, IIT Madras}{August 2012 - Present}
\titleAndGuide{Implementation of SSTA based timing algorithms on a GP-GPU}{Prof. Nitin Chandrachoodan}
\Point{
Exploring features and methods of implementing Statistical Static Timing Analysis Model (\textbf{SSTA})}
\Point{
Implementing the SSTA algorithm and accelerating it in NVidia Graphics processors using \textbf{CUDA}}
\Point{
Block-based SSTA parallel implementation while tackling limitations of the CUDA Architecture}

\datedsubsection{Canny Edge Detector - Image Processing on Hardware}{Jan - April 2012}
\titleAndGuide{Project: CAD for VLSI Design}{}
\Point{Implemented \textbf{Canny Edge Image Detection} algorithm}
\Point{\textbf{Pipelined} and \textbf{moving frame} implementation in \textbf{Verilog} for 128x128 px images}
\Point{
Built non-maximal suppression and thresholding blocks. Responsible for implementation and debugging in FPGA \textit{Team Size : 3}}
\Point{Simulated, synthesised and implemented in \textbf{Xilinx Spartan 6E FPGA}}

\section{Internships}
\datedsubsection{American Express India, Gurgaon}{May - July 2012}
%\hspace{2em}\textit{Risk Modelling of Credit Card Customers based on Initial Transactions} \hfill 
\Point{
Leveraged the \textbf{Longitudinal Behaviour} of \textbf{Early Transactions} towards predicting \textbf{Short \& Long Term Risk} 
- Model \textbf{2 times better risk} indication for early defaulters and \textbf{50 bp lower} for good customers than existing methods}
\Point{
Developed a system that extracts \textbf{risky/credit-worthy behaviour} and flags events across different markets worldwide}

\pagebreak

\datedsubsection{Ittiam Technologies Pvt Ltd, Bangalore}{May - June 2011}
%\hspace{2em}\textit{Low Density Parity Check(LDPC) models based on the IEEE 802.11n standard}\hfill 
\Point{
\textbf{Implemented} a model for \textbf{Low Density Parity Check}(LDPC) error correcting codes - based on the IEEE 802.11n standard}
\Point{
Disadvantages of the simple LDPC Encoder- Decoder realized. Proposed a suitable \textbf{Hardware structure} for implementing the LDPC Encoder and Decoder on chip}

\datedsubsection{Agent Technologies Pvt Ltd, Bangalore}{May - June 2010}
%\hspace{2em}\textit{License Plate Extraction for Residential/Retail Usage}\hfill 
\Point{
\textbf{Image Processing} using Basic Graph Theory. Modular library built in \texttt{C++} - used in extracting License Plate Details. Hit Rate of \textbf{70\%}}
\section{Course Work}
\begin{Course}
  \Point{Analog IC Design}
  \Point{RF IC Design}
  \Point{Digital IC Design}
  %\Point{Networks and Systems}
  \Point{VLSI Technology}
  %\Point{Digital Systems}
  %\Point{%EM Energy Conversion}
  %\Point{%Power Systems Practices}
  %\Point{Analog and DSP}
  % \Point{% EM Fields}
  \Point{CAD for VLSI}
  \Point{Analog Circuits}
  \Point{DSP Architectures and Embedded Systems}
  \Point{Mathematical Methods for Circuit Analysis}
  \Point{Digital Communication Systems}
  \Point{Machine Learning}
  %\Point{Solid State Devices}
  %\Point{Communication Systems}
  % \Point{% Control Engineering}
  %\Point{Device Modelling}
\end{Course}

\section{Labs}
\begin{Course}
  \Point{Analog Circuits Lab}
  \Point{Microprocessors Lab}
  \Point{Advanced Microelectronics Lab}
  \Point{Digital Circuits Lab}
  % \Point{% CAD Lab}
  %\Point{%Electromechanical Energy Lab}
\end{Course}
\section{Technical Skills}
\Point{Programming Languages : Python, C, SPICE, Verilog, \LaTeX}
\Point{Scientific Applications: MATLAB, Scilab}
% \Point{
% Operating Systems : Microsoft Windows (98, 2000, XP, Vista, 7), Linux distros like Ubuntu, Debian, Red Hat Linux
% }
\section{Positions Of Responsibility}
\datedsubsection{Core, Web And Mobile Operations, Shaastra}{May-October 2011}
\Point{
\textbf{Headed a team of 20 coordinators} to completely \textbf{rebuild} Shaastra Website in 2 months with \textbf{custom backend} - improved ease of use for Event Coordinators and Participants}
\Point{
\textbf{Laid the Foundations} for an in-house \textbf{Enterprise Resource Planning}(ERP) system for the \textbf{entire Shaastra Team}}
\Point{
\textbf{Initiated and Built} the Backend for \textbf{User Registration for Events Using BarCode} to speed up, and \textbf{document the participation registration} online for \textbf{analytics}; Built \textbf{Shaastra Android application} for the \textbf{first time}}

\subsection{Core, Election Engineering Team}
\Point{
\textbf{Led the Election Engineering Team} for the Student Body Elections 2011. Implemented a \textbf{Secure Web Based Voting} for a 5000-strong electorate across 20 polling stations}


\section{Extra Curricular Activities}
\Point{Simulation Championship- \textbf{Won $2^{nd}$ and $3^{rd}$} in Shaastra 2010, Shaastra 2009 respectively}
\Point{\textbf{Won $3^{rd}$ in Online Programming Contest}, Exebit 2009, technical festival by CS Dept of IIT Madras}
\Point{Participated in LM Solo Vocal Lit-Soc 2010. Trained in Vocal Carnatic Music}
\Point{Selected for \textbf{NSO Cricket} along with 20 others in total}

\end{document}

