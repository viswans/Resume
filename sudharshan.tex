\documentclass{mycv}
\author{Sudharshan Viswanathan}
\Address{320 Brown Street Apartment 517, West Lafayette Indiana 47906}
\PhoneNumber{+1 508 654-1337}
\Email{sudharshan.visu@gmail.com}

\begin{document}
\maketitle
\section{Professional Experience}
\datedsubsection{Quantitative Developer, Morgan Stanley India, Mumbai}{2013- 2015}
\Point{Implemented \textbf{Line Search minimization} methods separating callbacks from the minimization algorithm. Assisted an internship project in implementing Trust Region Strategies. Implementations to be used in Calibration use cases across the bank}
\Point{Refactored proprietary Date Rolling Library - Developed a \textbf{Stateless Referentially 
Transparent Immutable API} while removing the extant \textbf{Dependency} on \textbf{Singleton} pattern in C++.
Extend functionality to Java using \textbf{SWIG}}
\Point{Library Support and Maintenance - bug fixes and new additional features for the proprietary legacy Bond Analytics library while conforming 
to industry standards}
\Point{Integral part of the transition in the development environment of the team. Ensured backward compatibility of the build systems 
and responsible for setting up and testing of software releases}

\datedsubsection{Internship, American Express India, Gurgaon}{May - July 2012}
%\hspace{2em}\textit{Risk Modelling of Credit Card Customers based on Initial Transactions} \hfill 
\Point{Leveraged the \textbf{Longitudinal Behaviour} of early transactions towards predicting \textbf{Short \& Long Term Risk} 
- Model \textbf{2 times better risk} indication for early defaulters and \textbf{50 bps lower} for good customers than existing methods}
\Point{Developed a system that extracts \textbf{risky/credit-worthy behaviour} and flags events across different markets worldwide}

%\datedsubsection{Insternship, Ittiam Technologies Pvt Ltd, Bangalore}{May - June 2011}
%\hspace{2em}\textit{Low Density Parity Check(LDPC) models based on the IEEE 802.11n standard}\hfill 
%\Point{\textbf{Implemented} a model for \textbf{Low Density Parity Check}(LDPC) error correcting codes - based on the IEEE 802.11n standard}
%\Point{Disadvantages of the simple LDPC Encoder- Decoder realized. Proposed a suitable \textbf{Hardware structure} for implementing the LDPC Encoder and Decoder on chip}

%\datedsubsection{Agent Technologies Pvt Ltd, Bangalore}{May - June 2010}
%\hspace{2em}\textit{License Plate Extraction for Residential/Retail Usage}\hfill 
%\Point{\textbf{Image Processing} using Basic Graph Theory. Modular library built in \texttt{C++} - used in extracting License Plate Details. Hit Rate of \textbf{70\%}}

\section{Education \& Scholastic Achievements}
%\begin{EducationTable}
%	\EduDetails{\textbf{B.Tech}:Electrical Engineering \textbf{Minor}:Physics 
%   \newline \textbf{M.Tech}:VLSI and Microelectronics 
%   }{IIT Madras, Chennai}{\textbf {9.06} }{2013}
%\end{EducationTable}

\datedsubsection{Indian Institute of Technology Madras}{2008-2013}
\Point{\textbf{Major}: Electrical Engineering \textbf{Masters}: VLSI Design and Microelectronics \textbf{Minor}: Physics \textbf{CGPA: 9.06/10} }
\Point{\textbf{Technical Skills}. Prog. Languages - Advanced: C++, Verilog. Basic: Java, Scala, Python}
\Point{Winner of \textbf{TAU Contest}(Variation aware timing) 2012-13. Implemented a multi-threaded Statistical Timing Tool for multi-core CPU architecture. \textit{Team Size}: 2}
%\Point{Recipient of \textbf{Central Scheme Sector Scholarship} sponsored by the \textbf{MHRD} for excelling in XII CBSE examinations}
\Point{Selected to represent IIT Madras in the \textbf{Indo-German Winter Academy} conducted in December 2011 by IIT Delhi-FAU Erlangen. Presented a Technical Talk on \textbf{Resistive Memory Devices}}
\Point{Selected for the \textbf{International Olympiad in Informatics Training Camp}. Ranked in \textbf{top 25} at the Indian National Olympiad For Informatics(\textbf{INOI})}
%\Point{\textbf{First} in the Physics Talent Test in January 2008 conducted by The Physics Society, Chennai among \textbf{618} students}

\section{Projects}
\datedsubsection{Final Year Project}{2012-2013}
\Point{Statistical Static Timing Analysis Model(\textbf{SSTA}) based timing algorithms on a GP-GPU - 
Explored features, methods and drawbacks of different implementations of the algorithm}
\Point{Implemented and Anaylsed the Block Based SSTA algorithm on different MT environments:  \textbf{multi-core} and in \textbf{NVidia GPUs} using CUDA while
tackling constraints of CUDA architecture}

\datedsubsection{Course Projects}{2010-2012}
\Point{Digital VLSI: A Pipelined and Sliding-Window Canny Edge Detector on FPGA (\textit{CAD For VLSI Design}). 
Reciprocal and Square Root of Complex Numbers using spline interpolation on FPGA (\textit{DSP Architectures})}

\section{Positions Of Responsibility}
\datedsubsection{Core, Web And Mobile Operations, Shaastra}{May-October 2011}
\Point{Led a team of 20 coordinators to completely rebuild website for Shaastra, annual technical festival of IIT Madras - in 2 months with \textbf{custom backend} - improved ease of use for Event Coordinators and Participants}
\Point{Laid the foundations for an in-house \textbf{Enterprise Resource Planning}(ERP) system for the entire team of 300 students}

\subsection{Core, Election Engineering Team}
\Point{Led the Election Engineering team for the Student Body Elections 2011. Implemented a \textbf{Secure Web Based Voting} for a 5000-strong electorate across 20 polling stations}

\section{Extra Curricular Activities}
\Point{Integral part of Morgan Stanley Cricket team in Corporate Leagues.}
\Point{Trained in Vocal Carnatic Music. Participated in Solo Vocal competitions held in IIT Madras.}
%\Point{Won $3^{rd}$ in Online Programming Contest, Exebit 2009, technical festival by CS Dept of IIT Madras. 
%Simulation Championship- Won $2^{nd}$ and $3^{rd}$ in Shaastra 2010, Shaastra 2009 respectively}
\end{document}

